\documentclass[10pt,a4paper]{article}
\usepackage[utf8]{inputenc}
\usepackage[francais]{babel}
\usepackage[T1]{fontenc}
\usepackage{amsmath}
\usepackage{amsfonts}
\usepackage{amssymb}
\author{Paul CLAVIER}
\title{[Info525]Résolution}
\begin{document}
\maketitle
\newpage
\section{Déduction naturelle}
	\begin{itemize}
		\item règles et raisonnement 'naturels'
		\item automatisation (?)
	\end{itemize}
	\paragraph{informatique:}
	\begin{itemize}
		\item une seule règle
		\item simple
	\end{itemize}
	\paragraph{$\Rightarrow$} Le principe de résolution
	\begin{itemize}
		\item Démonstrations automatiques de Théorèmes
		\item Langage Prologue
	\end{itemize}
\section{Principes}
	\begin{itemize}
		\item les formules sont sous forme clausale
		\item utilisation de la FNC
	\end{itemize}
	\paragraph{Règles de résolution}:
		Soient deux clauses $c_1 = c'_1 \vee p$ et $c_2=c'_2 \vee \neg p$, \[c'_1\vee p, c'_2\vee\neg p \vdash c'_1\vee c'_2\]
	\paragraph{Définition}: La clause $c'_1\vee c'_2$ est appelée résolvant.
	\paragraph{Mise en œuvre}:
		\begin{description}
			\item[Rappel des théorèmes]: $A_1,\ldots, A_n \vdash B$ ssi $A_1,\ldots,A_n \models B$\\
				$A_1,\ldots,A_n \models B$ ssi $A_1 \wedge \ldots \wedge A_n \wedge \neg N \models B$
		\end{description}
		L'ensemble $\{A_1, \ldots, A_n, \neg B\}$ est inconsistant.
	\paragraph{Mise en pratique de la Résolution}
		\begin{itemize}
			\item Soit $E$ un ensemble de clauses
			\item On calcule tous les résolvants des paires de clauses
			\item On ajoute les résolvants à $E$
			\item Le procédé est répété jusqu’à trouver une paire de littéraux complémentaires.
			\item L'ensemble de clauses $E$ est alors inconsistant.
		\end{itemize}

\end{document}